In this section we prove that given our realistic parameter settings, fetching from memory will definitely become
an advantageous strategy compared to re-computation. We use sandwich graph from~\cite{corrigan2016balloon} as an
example. We take our ideas from the proof in~\cite{corrigan2016balloon}, especially, we are interested in the
\textit{big-spread} property and the \textit{everywhere-avoiding} property.

\subsection{Big spread lemma variant}
\paragraph{Room for improvement}
The \textit{big spread lemma} plays an important role in the security proof of balloon hashing algorithm \cite{corrigan2016balloon},
however, the theorem require that the subset being considered, $V\prime$ have the same size of the pebbles available by the adversary.
This is also reflected in Lemma~30, which bound the probability of the graph being too ``dense''. Recall in Lemma~30, for $\delta=3$,
every sandwich graph is an (m, m, n/16)-consecutively-avoiding graph, for $n_0 > 16$ and all $n_0 \leq m < n/6$ except with
probability $P_{consec}(n, d, n_0) \leq 2*8 \cdot d \cdot n \cdot 2^{-n_0/2}$; And for $\delta=7$, such stack of sandwich graph is
(n/64, n/32)-everywhere-avoiding graph when $n > 2^{11}$, except with probability $P_{every} \leq 128\cdot d\cdot 2^{-n/50}$.
Note that in the later part of the lemma, the size of $V^\prime$ is defined as $m=\frac{n}{2^{\omega+1}}$, and thus m grows with
n.

This limitation that the subset of $V$ begin fixed by the number of pebbles of the adversary is somehow inconvenient, in that to make
the probability of bad event small, one need the number of pebbles available grow, which is the contrary of natural instinct, as
the more pebble the adversary acquires, the easier the attack (re-computation) should be. Although the result in \cite{corrigan2016balloon}
is absolutely correct, as we can view the growth of m as the result of the growth of n, an if the subset $V^\prime$ being considered too
small, the variables sampled independently uniform at random from the last layer would have a bigger probability to being not so well-spread.
Still, we want to separate the size of subset ($m$) and the number of pebbles ($|V^\prime|$).

\begin{lemma}[Big Spread Lemma Variation]\label{lemma::bigspreadvar}
  For all positive integer $\delta \geq 3$, $\omega \geq 2$, $n_0$, and n, and for all positive integers m such that
  $2^\omega < m < 2^{2-\omega+\frac{2\omega}{\delta e}}$, a list of $\delta m$ elements sampled independently and uniformly at random from
  ${1,\ldots,n}$ is an ($n_0$, $n/2^\omega$)-well-spread set with probability at least
  $1-2^{(1-\omega)\frac{\delta}{2}m} \cdot (2^{\omega m}+2^{\omega+1}2^{\omega n_0})$
\end{lemma}

\begin{proof}
  The Strategy to bound the bad event that such property does not exist is the same as that used in \cite{corrigan2016balloon}.
  Let $R = (R_1, \dots, R_{\delta m})$ be integers sampled independently and uniformly at random from ${1,\dots,n}$, we want to
  prove that for all subset $S \subseteq R$ of size at most $n_0$, $spread_S(R) \geq n/2^\omega$. To do so, we first define a bad
  event B, and then show that bounding Pr[B] is enough to prove the lemma, then bound Pr[B] to complete the lemma.

  \paragraph{The Bad Event B}
  Write the integers in R in non-decreasing order as $X_1,\dots, X_{\delta m}$, then define $X_0 = 0$, $X_{\delta m+1} = n$.
  Let bad event B be the event that there exists a set $S^\prime \subseteq {X_1, \dots, X_{\delta m+1}}$ of size at most $(n_0+1)$,
  such that $\sum_{X_i\in S^\prime}{X_i - X_{i-1}} \geq (1-2^{-\omega})n$.

  Whenever there exists a set $S \subseteq R$ of size at most $n_0$ that cause $spread_S(R) < n/2^{\omega}$, then bad event B must occur.
  Assuming that such a bad set $S$ exists, construct a set $S^\prime = S \bigcup {X_{\delta m + 1}}$ of size at most $n_0 + 1$, Then we
  compute
  \begin{equation}
    \sum_{X_i \in S^\prime}{(X_i - X_{i-1})} = n - X_{\delta m} + \sum_{X_i \in S}{(X_i - X_{i-1})}
    = n - \sum_{X_i \not\in S}{(X_i - X_{i-1})}
  \end{equation}
  The last inequality holds because $X_{\delta m} = \sum_{X_i \in R}{(X_i - X_{i-1})}$. Now that $Spread_S(R) < n/2^\omega$, so we have
  bad event B occurs. Thus, Pr[$Spread_S(R)$] $\leq$ Pr[B], and therefore bounding a upper bound of bad event B is sufficient to prove the
  lemma.

  \paragraph{Strategy to bound Pr[B]}
  Let $D$ be a random variable denoting the number of distinct integers in the list of random integers $R$. For any fixed integer $D^* \in \{1,\dots,n\}$,
  we can write:
  \begin{align}
  Pr[B] &= Pr[B|D < d^*] \cdot Pr[D < d^*] + Pr[B|D \geq d^*] \cdot Pr[D \geq d^*] \\
  &\leq Pr[D < d^*] + Pr[B,D \geq d^*]
  \end{align}
  In the following proof, we take $d^* = \delta m/2$.

  \paragraph{Bounding $Pr[D < d^*$]}
  The probability that this event occurs is at most the probability when we throw $\delta m$ balls into $n$ bins, all the balls fall into a set of
  $\delta m/2$ bins.
  \begin{equation}
    Pr[D < d^*] \leq \binom{n}{d^*}(\frac{d^*}{n})^{\delta m} \leq (\frac{n \cdot e}{d^*})^{d^*}(\frac{d^*}{n})^{\delta m} = (\frac{d^*}{n})^{\delta m - d^*}e^{d^*}
  \end{equation}
  By the hypothesis of the lemma, $m < 2^{(1-\omega + \frac{2\omega}{\delta})}\frac{2n}{\delta e}$. Therefore, we have
  \begin{equation}
    Pr[D < d^*] \leq (\frac{\delta me}{2n})^{\delta m/2} \leq 2^{[(1-\omega)\frac{\delta}{2}+\omega]m}
  \end{equation}

  \paragraph{Bounding $Pr[B, D \geq d^*]$}
  First I would like to re-state the lemma~8 in the original work \cite{corrigan2016balloon}, because this lemma is useful in the proof of big spread lemma.
  \begin{lemma}\label{lemma::randommap}
    Let $(R_1,\dots,R_d)$ be random variables respecting integers sampled uniformly, but without replacement, from $\{1,\dots,n\}$. Write the $R$s in ascending
    order as $(Y_1,\dots,Y_d)$, and let $Y_0 = 0$, and $Y_{d+1} = n$. Next define $L = (L_1,\dots,L_{d+1})$, where $L_i = Y_i - Y_{i-1}$. Then for all functions
    $f:\mathbb{Z}^{d+1} \rightarrow \{0,1\}$, and for all permutations $\pi$ on $d+1$ elements,
    \begin{equation}
      Pr[f(L_1,\dots,L{d+1}) = 1] = Pr[f(L_{\pi(1)},\dots,L_{\pi(d+1)}) = 1].
    \end{equation}
  \end{lemma}

  Particularly, this lemma shows that if we define a function that maps a list of segments (separated from $\{1,\dots,n\}$ by the $\delta m$ random integers)
  in to $\{0,1\}$, such that
  if bad event $B$ happens, then it is one. Using this lemma, we can define a permutation on those integers that if it appears in a bad set, then we shift it with
  the first $n_0+1$ integers in the list, and thus we can only focus on the first few elements on the list without changing the result. As lemma~\ref{lemma::randommap}
  require the list to be mutually distinct, we have to transform the random list in to a distinct one first.

  \textit{Step~1:} Bad event $B$ occurs whenever there is a subset $S^\prime \subseteq X = (X_1,\dots,X_{\delta m+1})$ of size at most $(n_0 + 1)$, such that
  $\sum_{X_i \in S^\prime}{(X_i - X_{i-1})} \geq (1-2^{-\omega})n$. Whenever bad event B occurs, there is also a subset $S_{dist}^\prime$ of distinct integers that
  also satisfy the sum equation, and also has the size $(n_0 + 1)$. (This is because if duplicate exists, we can always delete duplicates, without changing the sum,
  and then add new integers to make the sum even bigger.)
  thus we can use lemma~\ref{lemma::randommap} to simplify calculation.

  \textit{Step~2:} we consider the case there is explicitly $d$ distinct integers in the list $R$, and then use union bound to deduce the final probability.
  Write the $d$ distinct integers in $R$ is ascending order as $Y = (Y_1,\dots,Y_d)$. We want to bound the probability that some subset $Y_{dist}^\prime$ of size
  $n_0$ is bad. (Note that the only change from the original lemma is that I changed $m$ into $n_0$.) Then define $L_i$ as in lemma~\ref{lemma::randommap}.
  Let the set of indices $I \subseteq \{1,\dots,d+1\}$ of size $n_0 + 1$ be the set of index of bad subset. That is, $S_{dist}^\prime = \{Y_i|i \in I\}$.
  Then define the permutation $\pi: \mathbb{Z}^{d+1}\rightarrow\mathbb{Z}^{d+1}$, such that if $i \in I$, then $L_i$ appears in the first $n_0 + 1$ elements
  of $(L_{\pi(1)},\dots,L_{\pi(d+1)})$. Define function $f:\mathbb{Z}^{d+1}\rightarrow \{0,1\}$ such that if its first $n_0 + 1$ arguments sums to
  at least $(1-2^{-\omega})n$, returns 1 and it returns 0 otherwise. Therefore, we have:
  \begin{align}
    Pr[\sum_{i \in I}{L_i} \geq (1 - 2^{-\omega})n|D = d] &= Pr[f(L_{\pi(1)},\dots,L_{\pi(d+1)}) = 1|D = d] \\
    &= Pr[f(L_1,\dots,L{d+1}) = 1|D = d] \\
    &= Pr[(L_1 + \dots + L_{n_0+1}) \geq (1-2^{-\omega})n|D = d]
  \end{align}

  \textit{Step~3:} the event defined above can only happen when the other $d - (n_0 + 1)$ integers falls into the right most $2^{-\omega}n$ bins, and then
  we have:
  \begin{equation}
    Pr[(L_1 + \dots + L_{n_0+1}) \geq (1-2^{-\omega})n|D = d] \leq (\frac{1}{2})^{\omega(d-n_0-1)}
  \end{equation}
  Above is the probability that a single set is bad, we then sum it over all $\binom{d+1}{n_0+1}$ possible size $n_0 + 1$ subsets, to get:
  \begin{align}
    Pr[B|D = d] &\leq \binom{d+1}{n_0+1}(\frac{1}{2})^{\omega(d-n_0-1)} \\
    &\leq 2^{(1-\omega)d + \omega n_0 + \omega + 1}
  \end{align}
  Equipped with $Pr[B|D = d]$, we can then proceed to calculate $Pr[B, D \geq d^*]$. Note when $\omega > 1$, $Pr[B|D = d]$ is non-increasing
  in d. And therefore we have:
  \begin{align}
    Pr[B, D \geq d^*] &\leq 2^{(1-\omega)d^* + \omega n_0 + \omega + 1} \cdot \sum_{d = d^*}^{\delta m + 1}{Pr[D = d]} \\
    &\leq 2^{1+\omega} \cdot 2^{(1-\omega)\frac{\delta m}{2} + \omega n_0}
  \end{align}

  \paragraph{Completing the proof}
  By the calculation above, we have
  \begin{align}
    Pr[B] &\leq Pr[D < d^*] + Pr[B, D \geq d^*] \\
    &\leq 2^{(1-\omega)\frac{\delta m}{2}} \cdot (2^{\omega m} + 2^{1+(n_0+1)\omega})
  \end{align}
\end{proof}



After proving the big spread lemma variant, we are interested in using it to deduce everywhere avoiding property of the
sandwich graph, with the size of the subset $V^\prime \subset V$ and the pebble constraint independent.

\subsection{Everywhere avoiding property}
\paragraph{An analysis of original strategy}
In the original work \cite{corrigan2016balloon}, the author proves a \textit{consecutive avoiding property} over all possible size of $m$, and
then for m of fixed size ($n/64$), proves a tighter \textit{everywhere avoiding property}. The reason of combining these two property is that
the first one allows the number pebbles to be arbitrary within a given range. In particular, one can consider $m$ to have the same number of the
pebbles the adversary owns. In order to prove a tighter bound on $S \cdot T$ (like in part~b of Lemma~31), one needs to first apply consecutive
avoiding property and then use the everywhere avoiding property. What we want to do is to acquire the probability that given at most $n_0$ pebbles,
any $m$ pebbles on a standard sandwich graph is an $(n_0, m, 2/2^\omega)$ avoiding set.

\begin{lemma}[Everywhere Avoiding]\label{lemma::everyAvoid}
  Let $G = (U \bigcup V, E)$, be a $\delta$-random sandwich graph on 2n vertices. Then for all positive integer $m$ that satisfy the assumption
  of lemma~\ref{lemma::bigspreadvar}, and $n_0$ such that $n_0$ is no bigger than the upper bound of $m$, we have that given $n_0$ pebbles at most,
  every subset $V^\prime \subseteq V$ of size $m$, is a $(n_0, n/2^\omega)$ avoiding set with probability at least
  $1 - (\frac{n}{m})^m \cdot 2^{(1-\omega)\frac{\delta m}{2}} \cdot (2^{\omega m} + 2^{1 + (n_0 + 1) \omega})$
\end{lemma}

\begin{proof}
  Fix the size of $n_0$ and $m$, all we need to do is to apply union bound to all possible choice of $V^\prime$, which has $\binom{n}{m}$ in total.
  Apply this inequality to simply the bound:
  \begin{equation}
    \binom{n}{m} = \frac{n \times (n-1) \times \dots \times (n-m+1)}{m \times (m-1) \times \dots 1} \leq
    \frac{n^m}{m^m} = (\frac{n}{m})^m
  \end{equation}
  And thus by lemma~\ref{lemma::bigspreadvar}, we have the probability that a single bad event occurs is at most
  $2^{(1-\omega)\frac{\delta}{2}m} \cdot (2^{\omega m}+2^{\omega+1}2^{\omega n_0})$, and thus the probability that
  for all choice of $V^\prime$, the bad event occurs is at most
  $(\frac{n}{m})^m \cdot 2^{(1-\omega)\frac{\delta m}{2}} \cdot (2^{\omega m} + 2^{1 + (n_0 + 1) \omega})$.
  And thus the lemma is proved.
\end{proof}

\subsection{A bound on the pebbling moves of random sandwich graph}
\paragraph{Bounding pebbling moves}
And now for the sake of our own objective, we would like to calculate the bound of re-computation of a subset of vertices on any layer. To do this, we
need to first compute the moves needed to pebble a subset of vertices. First define a variable to denote the pebbling moves needed to pebble a subset of
$m$ vertices on the condition that except on those $m$ vertices, there can be at most $n_0$ pebbles on any vertices of the graph.

\begin{definition}\label{def::Tr}
  Let $G$ be a stack of random sandwich graphs. Then define $T_r$ be the pebbling moves needed to pebble a subset $V^\prime$ of vertices on layer $r$, which has
  not received any pebble on the beginning of the moves, and at the end of the moves, the topologically last one receives a pebble. The condition is that there
  can only be at most $n_0$ pebbles on the graph.
\end{definition}

Now that we have defined the amount to calculate, we can use induction reasoning to calculate $T_r$.

\begin{lemma}\label{lemma::rbound}
  Given the average case that $T_0 = \frac{m+n}{2}$, we have:
  \begin{equation}
    T_r \geq -\frac{m}{\frac{n}{m2^\omega} - 1} + (\frac{n}{m2^\omega})^r(\frac{m+n}{2}+\frac{m}{\frac{n}{m2^\omega} - 1})
  \end{equation}
  where $0 \leq r \leq d$, $d$ is the layer of the stack of random sandwich graph.
\end{lemma}
\begin{proof}
  By the definition of $T_r$, we have that $T_r \geq m + \frac{n}{m2^\omega}T_{r-1}$. And thus we have:
  \begin{align}
    T_r &\geq m + \frac{n}{m2^\omega}T_{r-1}\\
    T_r + \frac{m}{\frac{n}{m2^\omega} - 1} &\geq \frac{n}{m2^\omega}(T_{r-1} + \frac{m}{\frac{n}{m2^\omega} - 1})\\
  \end{align}
  If we define $Y_r = T_r + \frac{m}{\frac{n}{m2^\omega} - 1}$, and change the inequality into equality, we can have
  \begin{equation}
    Y_r = (\frac{n}{m2^\omega})^r \cdot Y_0
  \end{equation}
  And thus $T_r \leq -\frac{m}{\frac{n}{m2^\omega} - 1} + (\frac{n}{m2^\omega})^r(\frac{m+n}{2}+\frac{m}{\frac{n}{m2^\omega} - 1})$.
\end{proof}

Then we define the lower bound of re-computation needed in the process discussed above.
\begin{definition}
  Let Recompute(r) be the pebbling moves needed in the process to pebble m vertices on layer r. More specifically, the pebbling moves needed to
  pebble the direct predecessors of those m vertices.
\end{definition}

By the definition of $Recompute(r)$ and lemma~\ref{lemma::rbound}, one can see that $Recompute(r) \geq -m + T_r$, and it grows exponentially with r.

\subsection{A comparison on re-computation and DRAM fetching}
\paragraph{Parameter setting and result}
In the following comparison section, we take the block size, which the output length of the underlying cryptographic hashing function to be 256~bit,
and the block in one layer of the graph to be $2^22$\footnote{The authors of Balloon Hashing recommended the space parameter to be 256MB for authentication servers.}
. We take $\omega = 3$, and the ratio of the latency of fetching a DRAM block over the ratio of
the time needed to output a block from Random Oracle, $c = 1000$.

The result of the ratio of naive DRAM algorithm (only fetch from memory, never re-compute), and
the lower bound proved in lemma~\ref{lemma::rbound} are given together with the depth of layer in
figure~\ref{table::recompDramRatio}. We can see from the ratio that DRAM fetching becomes advantageous once the
graph gets deeper than 5 layers. However, this is a loose bound, as optimal strategy might combine these two strategies,
using only the optimal moves. An intuitive idea is that when other parameters other than the depth of the graph are fixed,
those re-computation optimal moves can only occur at the first few layers, and thus given the fact that $r$ is growing,
the combination boost is bounded by a constant.

\begin{table}
  \centering
  \begin{tabular}{|c|c|}
     \hline
     % after \\: \hline or \cline{col1-col2} \cline{col3-col4} ...
     Layer & Ratio \\ \hline
     1 & 0.01 \\
     2 & 0.04 \\
     3 & 0.15 \\
     4 & 0.62 \\
     5 & 2.46 \\
     \hline
  \end{tabular}
  \caption{The ratio of re-computation over DRAM fetching}\label{table::recompDramRatio}
\end{table}

\paragraph{A bound on the memory hardness of balloon hashing}
After the intuition that as the depth goes deeper, algorithm using DRAM fetching will certainly be advantageous, we are interested in
finding a lower bound on the $c_0$ defined in definition~\ref{def::memoryHard}. If we constrain the computational entity to only computing
$m$ adjacent blocks once at a time, then the problem would be trivial, as lemma~\ref{lemma::rbound} and simulation result in table~\ref{table::recompDramRatio}
shows that as $r$ gets greater, DRAM fetching will certainly be the optimal choice. Also, in shallower layers re-computation only takes up a
small portion of total time (they cannot exceed $c \cdot \delta \cdot m$), then we have $c_0 > 1 - r_0/r$, where $r_0$ is the number of deepest layer in
the group of shallow layers. Now the tricky part is that how to remove the constraint.

% Consider an algorithm not using DRAM computing on a random sandwich graph, then there always exists a window of size $m$ not computed ahead of
% the latest computed node (but for the last $m$ nodes). And the computation can be considered to be computing the first node in the window.
% As when doing this, no nodes exist in the window, and the condition of such actions are somewhat identical.
% I feel very uncertain about this reasoning
%
% This give rise to the idea that we can consider the time required to compute each blocks in the subset $V^\prime$ of size $m$ defined in
% lemma~\ref{lemma::everyAvoid} to be identical. As the fetching strategy is scalable in nature, we can shrink $m$ into 1 in this way.
% And thus any entity computing on this graph must choose between two strategies. And our conclusion that $c_0 > 1 - r_0/r$ suffices.

Naturally, if we can bound that for any optimal strategy computing the $V^\prime$ group of size $m$, there must exist a constant portion of blocks that
are fetched from DRAM, then the memory hard property from definition~\ref{def::memoryHard} is satisfied.

\begin{theorem}\label{theorem::memoryHardLowerBound}
  Fix a random sandwich graph $G$ with $r$ layers and $n$ nodes per layer, and given $m$ and $n_0$ as defined in~\ref{lemma::everyAvoid}, the
  memory hardness parameter defined in definition~\ref{def::memoryHard} satisfies that
  % the greater sign here is not accurate, need to be improved
  % this should be approximation
  %
  \begin{equation}
    c_0 > \frac{(m - \frac{n_0}{\delta}) \cdot \delta c}{\frac{n_0}{\delta} + (m - \frac{n_0}{\delta}) \cdot (\delta + 1)c}
  \end{equation}
\end{theorem}

Before we prove theorem~\ref{theorem::memoryHardLowerBound}, a little explanation here is necessary. The idea behind this is that
when considering pebbling a group of size $m$ haven't been pebbled, with large probability, only those nodes with parents fully
pebbled can be advantageous in the case of ``red pebble'' solution. And indeed, if this is the case ``red solution'' would
certainly be better than ``blue solution''. But due to the limitation on the total pebbles $n_0$, there can only exist a constant
portion of such ``red solutions'', leaving the predecessor of the rest to be fetched from memory. And thus the memory hard property
from definition~\ref{def::memoryHard} is satisfied.

The following lemma proves that if the predecessor of a node is partially pebbled, then using ``red solution'' is takes exponential
time as the depth $r$ goes further.

\begin{lemma}\label{lemma::quality}
  Let $G$ be the random sandwich graph defined in theorem~\ref{theorem::memoryHardLowerBound}, if a node in layer $r$
  is not pebbled, and there exists at least one unpebbled parent of this node, then with probability at least:
  \begin{equation}
    Pr = \frac{\binom{n_0 - 1}{n - m}}{\binom{n_0 - 1}{n - 1}} \cdot \frac{n - m + 1}{n}
  \end{equation}
  the moves needed to pebble this node is at least $T_{r - 1}$ as defined in definition~\ref{def::Tr}.
\end{lemma}

\begin{proof}
  Let $x$ denote the unpebbled parent, and if at least $m$ consecutively previous nodes are unpebbled, then to pebble $x$
  has the same effect as pebbling all those $m$ nodes. Note that in this case we can choose arbitrary $m$, so long as the
  requirement from lemma~\ref{lemma::bigspreadvar} is satisfied. (if $m$ is too big, the event would be unlikely) The rest
  could be considered as fix one bin, and drop $n_0$ balls into the rest $n - 1$ bins. Note that we ignored the corner
  case where $x$ resides in the first $m - 1$ nodes of layer $r - 1$. Other than the corner case, fix the position of $x$,
  and the probability is $\binom{n_0 - 1}{n - m}/\binom{n_0 - 1}{n - 1}$. Taking all the probability in the corner
  case as zero would get the final result.

  As shown, with high probability the previous $m$ nodes of $x$ is unpebbled. And thus we can use lemma~\ref{lemma::rbound}
  to deduce that to pebble $x$, at least $T_{r - 1}$ moves are required, which proves the lemma.
\end{proof}

Now that we see an optimal pebbling strategy would not waste pebbles on nodes with partially pebbled parents. And
therefore the optimal strategy we are talking are left with two cases:

\begin{itemize}
  \item[Red] all parents of this node is fully pebbled, and it takes one round to pebble this node
  \item[Blue] any parent of this node is not pebbled, and it takes $\delta c + 1$ rounds to pebble this node
\end{itemize}

Given there are only $n_0$ red pebbles, one can easily calculate the portion of DRAM fetching moves used in pebbling
these $m$ nodes, and thus the memory hardness of the function.

\begin{proof}[Proof of theorem~\ref{theorem::memoryHardLowerBound}]
  From lemma~\ref{lemma::quality} we can deduce that in a group of $m$ adjacent nodes, $n_0 / \delta$ of them are pebbled
  directly, and the rest is pebbled using the blue pebble moves. And thus the total rounds is
  \begin{equation}
    T = \frac{n_0}{\delta} + (m - \frac{n_0}{\delta}) \cdot (\delta + 1) c
  \end{equation}
  where $(m - \frac{n_0}{\delta}) \cdot \delta c$ rounds are used to fetch blocks from DRAM. After $r$ gets deep enough,
  the majority of the $m$ adjacent nodes will satisfy the property. And this completes the proof.
\end{proof}
