\paragraph{Room for improvement}
The \textit{big spread lemma} plays an important role in the security proof of balloon hashing algorithm \cite{corrigan2016balloon},
however, the theorem require that the subset being considered, $V\prime$ have the same size of the pebbles available by the adversary.
This is also reflected in Lemma~30, which bound the probability of the graph being too ``dense''. Recall in Lemma~30, for $\delta=3$,
every sandwich graph is an (m, m, n/16)-consecutively-avoiding graph, for $n_0 > 16$ and all $n_0 \leq m < n/6$ except with
probability $P_{consec}(n, d, n_0) \leq 2*8 \cdot d \cdot n \cdot 2^{-n_0/2}$; And for $\delta=7$, such stack of sandwich graph is
(n/64, n/32)-everywhere-avoiding graph when $n > 2^{11}$, except with probability $P_{every} \leq 128\cdot d\cdot 2^{-n/50}$.
Note that in the later part of the lemma, the size of $V^\prime$ is defined as $m=\frac{n}{2^{\omega+1}}$, and thus m grows with
n.

This limitation that the subset of $V$ begin fixed by the number of pebbles of the adversary is somehow inconvenient, in that to make
the probability of bad event small, one need the number of pebbles available grow, which is the contrary of natural instinct, as
the more pebble the adversary acquires, the easier the attack (re-computation) should be. Although the result in \cite{corrigan2016balloon}
is absolutely correct, as we can view the growth of m as the result of the growth of n, an if the subset $V^\prime$ being considered too
small, the variables sampled independently uniform at random from the last layer would have a bigger probability to being not so well-spread.
Still, we want to separate the size of subset ($m$) and the number of pebbles ($|V^\prime|$).

\begin{lemma}[Big Spread Lemma Variation]\label{lemma::bigspreadvar}
  For all positive integer $\delta \geq 3$, $\omega \geq 2$, $n_0$, and n, and for all positive integers m such that
  $2^\omega < m < 2^{2-\omega+\frac{2\omega}{\delta e}}$, a list of $\delta m$ elements sampled independently and uniformly at random from
  ${1,\ldots,n}$ is an ($n_0$, $n/2^\omega$)-well-spread set with probability at least
  $1-2^{(1-\omega)\frac{\delta}{2}m} \cdot (2^{\omega m}+2^{\omega+1}2^{\omega n_0})$
\end{lemma}

\begin{proof}
  The Strategy to bound the bad event that such property does not exist is the same as that used in \cite{corrigan2016balloon}.
  Let $R = (R_1, \dots, R_{\delta m})$ be integers sampled independently and uniformly at random from ${1,\dots,n}$, we want to
  prove that for all subset $S \subseteq R$ of size at most $n_0$, $spread_S(R) \geq n/2^\omega$. To do so, we first define a bad
  event B, and then show that bounding Pr[B] is enough to prove the lemma, then bound Pr[B] to complete the lemma.

  \paragraph{The Bad Event B}
  Write the integers in R in non-decreasing order as $X_1,\dots, X_{\delta m}$, then define $X_0 = 0$, $X_{\delta m+1} = n$.
  Let bad event B be the event that there exists a set $S^\prime \subseteq {X_1, \dots, X_{\delta m+1}}$ of size at most $(n_0+1)$,
  such that $\sum_{X_i\in S^\prime}{X_i - X_{i-1}} \geq (1-2^{-\omega})n$.

  Whenever there exists a set $S \subseteq R$ of size at most $n_0$ that cause $spread_S(R) < n/2^{\omega}$, then bad event B must occur.
  Assuming that such a bad set $S$ exists, construct a set $S^\prime = S \bigcup {X_{\delta m + 1}}$ of size at most $n_0 + 1$, Then we
  compute
  \begin{equation}
    \sum_{X_i \in S^\prime}{(X_i - X_{i-1})} = n - X_{\delta m} + \sum_{X_i \in S}{(X_i - X_{i-1})}
    = n - \sum_{X_i \not\in S}{(X_i - X_{i-1})}
  \end{equation}
  The last inequality holds because $X_{\delta m} = \sum_{X_i \in R}{(X_i - X_{i-1})}$. Now that $Spread_S(R) < n/2^\omega$, so we have
  bad event B occurs. Thus, Pr[$Spread_S(R)$] $\leq$ Pr[B], and therefore bounding a upper bound of bad event B is sufficient to prove the
  lemma.

  \paragraph{Strategy to bound Pr[B]}
  Let $D$ be a random variable denoting the number of distinct integers in the list of random integers $R$. For any fixed integer $D^* \in \{1,\dots,n\}$,
  we can write:
  \begin{align}
  Pr[B] &= Pr[B|D < d^*] \cdot Pr[D < d^*] + Pr[B|D \geq d^*] \cdot Pr[D \geq d^*] \\
  &\leq Pr[D < d^*] + Pr[B,D \geq d^*]
  \end{align}
  In the following proof, we take $d^* = \delta m/2$.

  \paragraph{Bounding $Pr[D < d^*$]}
  The probability that this event occurs is at most the probability when we throw $\delta m$ balls into $n$ bins, all the balls fall into a set of
  $\delta m/2$ bins.
  \begin{equation}
    Pr[D < d^*] \leq \binom{n}{d^*}(\frac{d^*}{n})^{\delta m} \leq (\frac{n \cdot e}{d^*})^{d^*}(\frac{d^*}{n})^{\delta m} = (\frac{d^*}{n})^{\delta m - d^*}e^{d^*}
  \end{equation}
  By the hypothesis of the lemma, $m < 2^{(1-\omega + \frac{2\omega}{\delta})}\frac{2n}{\delta e}$. Therefore, we have
  \begin{equation}
    Pr[D < d^*] \leq (\frac{\delta me}{2n})^{\delta m/2} \leq 2^{[(1-\omega)\frac{\delta}{2}+\omega]m}
  \end{equation}

  \paragraph{Bounding $Pr[B, D \geq d^*]$}
  First I would like to re-state the lemma~8 in the original work \cite{corrigan2016balloon}, because this lemma is useful in the proof of big spread lemma.
  \begin{lemma}\label{lemma::randommap}
    Let $(R_1,\dots,R_d)$ be random variables respecting integers sampled uniformly, but without replacement, from $\{1,\dots,n\}$. Write the $R$s in ascending
    order as $(Y_1,\dots,Y_d)$, and let $Y_0 = 0$, and $Y_{d+1} = n$. Next define $L = (L_1,\dots,L_{d+1})$, where $L_i = Y_i - Y_{i-1}$. Then for all functions
    $f:\mathbb{Z}^{d+1} \rightarrow \{0,1\}$, and for all permutations $\pi$ on $d+1$ elements,
    \begin{equation}
      Pr[f(L_1,\dots,L{d+1}) = 1] = Pr[f(L_{\pi(1)},\dots,L_{\pi(d+1)}) = 1].
    \end{equation}
  \end{lemma}

  Particularly, this lemma shows that if we define a function that maps a list of segments (separated from $\{1,\dots,n\}$ by the $\delta m$ random integers)
  in to $\{0,1\}$, such that
  if bad event $B$ happens, then it is one. Using this lemma, we can define a permutation on those integers that if it appears in a bad set, then we shift it with
  the first $n_0+1$ integers in the list, and thus we can only focus on the first few elements on the list without changing the result. As lemma~\ref{lemma::randommap}
  require the list to be mutually distinct, we have to transform the random list in to a distinct one first.

  \textit{Step~1:} Bad event $B$ occurs whenever there is a subset $S^\prime \subseteq X = (X_1,\dots,X_{\delta m+1})$ of size at most $(n_0 + 1)$, such that
  $\sum_{X_i \in S^\prime}{(X_i - X_{i-1})} \geq (1-2^{-\omega})n$. Whenever bad event B occurs, there is also a subset $S_{dist}^\prime$ of distinct integers that
  also satisfy the sum equation, and also has the size $(n_0 + 1)$. (This is because if duplicate exists, we can always delete duplicates, without changing the sum,
  and then add new integers to make the sum even bigger.)
  thus we can use lemma~\ref{lemma::randommap} to simplify calculation.

  \textit{Step~2:} we consider the case there is explicitly $d$ distinct integers in the list $R$, and then use union bound to deduce the final probability.
  Write the $d$ distinct integers in $R$ is ascending order as $Y = (Y_1,\dots,Y_d)$. We want to bound the probability that some subset $Y_{dist}^\prime$ of size
  $n_0$ is bad. (Note that the only change from the original lemma is that I changed $m$ into $n_0$.) Then define $L_i$ as in lemma~\ref{lemma::randommap}.
  Let the set of indices $I \subseteq \{1,\dots,d+1\}$ of size $n_0 + 1$ be the set of index of bad subset. That is, $S_{dist}^\prime = \{Y_i|i \in I\}$.
  Then define the permutation $\pi: \mathbb{Z}^{d+1}\rightarrow\mathbb{Z}^{d+1}$, such that if $i \in I$, then $L_i$ appears in the first $n_0 + 1$ elements
  of $(L_{\pi(1)},\dots,L_{\pi(d+1)})$. Define function $f:\mathbb{Z}^{d+1}\rightarrow \{0,1\}$ such that if its first $n_0 + 1$ arguments sums to
  at least $(1-2^{-\omega})n$, returns 1 and it returns 0 otherwise. Therefore, we have:
  \begin{align}
    Pr[\sum_{i \in I}{L_i} \geq (1 - 2^{-\omega})n|D = d] &= Pr[f(L_{\pi(1)},\dots,L_{\pi(d+1)}) = 1|D = d] \\
    &= Pr[f(L_1,\dots,L{d+1}) = 1|D = d] \\
    &= Pr[(L_1 + \dots + L_{n_0+1}) \geq (1-2^{-\omega})n|D = d]
  \end{align}

  \textit{Step~3:} the event defined above can only happen when the other $d - (n_0 + 1)$ integers falls into the right most $2^{-\omega}n$ bins, and then
  we have:
  \begin{equation}
    Pr[(L_1 + \dots + L_{n_0+1}) \geq (1-2^{-\omega})n|D = d] \leq (\frac{1}{2})^{\omega(d-n_0-1)}
  \end{equation}
  Above is the probability that a single set is bad, we then sum it over all $\binom{d+1}{n_0+1}$ possible size $n_0 + 1$ subsets, to get:
  \begin{align}
    Pr[B|D = d] &\leq \binom{d+1}{n_0+1}(\frac{1}{2})^{\omega(d-n_0-1)} \\
    &\leq 2^{(1-\omega)d + \omega n_0 + \omega + 1}
  \end{align}
  Equipped with $Pr[B|D = d]$, we can then proceed to calculate $Pr[B, D \geq d^*]$. Note when $\omega > 1$, $Pr[B|D = d]$ is non-increasing
  in d. And therefore we have:
  \begin{align}
    Pr[B, D \geq d^*] &\leq 2^{(1-\omega)d^* + \omega n_0 + \omega + 1} \cdot \sum_{d = d^*}^{\delta m + 1}{Pr[D = d]} \\
    &\leq 2^{1+\omega} \cdot 2^{(1-\omega)\frac{\delta m}{2} + \omega n_0}
  \end{align}

  \paragraph{Completing the proof}
  By the calculation above, we have
  \begin{align}
    Pr[B] &\leq Pr[D < d^*] + Pr[B, D \geq d^*] \\
    &\leq 2^{(1-\omega)\frac{\delta m}{2}} \cdot (2^{\omega m} + 2^{1+(n_0+1)\omega})
  \end{align}
\end{proof}

