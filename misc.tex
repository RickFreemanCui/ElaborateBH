% @main.tex section: introduction paragraph: our idea
Other works in this field mainly focus on analysing the time-space tradeoff of memory hard functions. This is a trivial
idea as if this is bounded, any adversary with limited memory has bounded execution time. However, our concern is that
due to fact that slow memory of large volume like synchronous DRAM can be used by application specific hardware designers
as in general purpose hardware. And thus we cannot bound the adversary to using only those fast memory. Our idea is to
prove that in some algorithms (Balloon hashing in particular), the computation needed to generate a block grows with its
`depth'. But the time needed to fetch a block of data from memory is fixed. This means that if the algorithm is configured
well, the main proportion of the execution time of the function will be spent on fetching blocks from memory. And because
the speed of DRAM is the same for general purpose computers and application-specific hardware, the goal to bring adversary
and honest users to the same level can be achieved.

% @main.tex section: model and definition
In previous works, memory hard functions are mainly studied via a ``pebbling reduction'', i.e. a mapping from
arbitrary effective computational strategy of a given function to a corresponding legal pebbling strategy on its
underlying data-dependency graph, and vise versa. We will continue this scheme, but as we have introduced slow
memory, we have to add some new rules to the pebbling games.