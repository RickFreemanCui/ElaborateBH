In this section we formally describe our model and prepare for the proof of our results in section~\ref{sec::sandwich}. The intuition behind
our model is that the common means in constructing a memory-hard function is by carefully constructing the underlying data dependency graph,
to make re-computation (which is unavoidable since by assumption custom circuit has only limited memory) takes very long time, as the function
runs towards the end. But those values re-computed has already been computed, and fetching a block of memory with arbitrary address from DRAM
takes constant time. This gives rise of the following intuition: If we can combine DRAM and custom circuit together, when such re-computation
becomes not advantageous compared to memory fetching, then can we improve the performance of custom circuit? After survey of current VLSI and
DRAM technology, we believe our model is useful given real parameter settings.

\paragraph{Constant definition}
We will use the following constant throughout the analysis. We use $T$ to denote the absolute time needed to for a computational entity to acquire
the hash of a given input. In the Random Oracle setting, this is the time required to access a particular random oracle call. We use $c \cdot T$
to denote the time required to fetch a block of memory (equals to the size of RO output), and thus $c$ is the ratio representing the fetch time over
the hash time\footnote{$c$ is approximately 1000}.
We will see later that this value correspond to the rounds that a legal pebbling must wait after making a pebble fetch move.

\subsection{Pebbling in DRAM model}

In this section we define rules of sequential pebbling game in our elaborated DRAM model.

\begin{definition}\label{def::dramPebblingRules}
  Let $G = (V, E)$ be a directed acyclic graph, a legal pebbling on $G$ given $S$ pebbles and target $T \subset V$
  is a sequence of moves, measured in rounds.
  Each move must be legal according to the following rules:

  \begin{enumerate}
    \item Put a red pebble on source node
    \item Put a red pebble on a node only if all of its parents have red node or blue node in the previous round
    \item Put a blue pebble on any nodes previously pebbled
    \item Remove a pebble from any node
  \end{enumerate}

  Additional constraints are at any time, there can be only at most $S$ red pebbles, and blue pebbles are removed automatically one round
  after placement. Move~1, 2, and 4 takes one round and move~3 takes $c$ rounds. A legal game is valid only if it begins with no node on the
  graph and end with all nodes in $T$ has been pebbled at least once.
\end{definition}

Note that in our definition of pebbling games, only one rule is added compared to the pebbling rules in~\cite{alwen2015high,corrigan2016balloon},
that is the blue pebble placement rule. This is an simplified abstraction of DRAM fetching in the real world. By our assumption, a block
can only be used immediately after being fetched, then it is lost. Of course in the real world, certain cache method is most likely applied
to save some time when, say, one block is repeatedly fetched. But as we will later show, this somewhat naive abstraction will save considerably
amount of time given real parameter setting.

\subsection{DRAM computational model}
In this section we define the theoretical computational machine on which algorithm with DRAM fetching is executed.

\begin{definition}\label{def::dramComputationalModel}
  Let $\mathcal{M}$ be a computational machine with available memory blocks of number $\sigma$ and an instruction set as follows:

  \begin{itemize}
    \item \textbf{HASH}: compute the hash\footnote{standard cryptographic hash function like SHA-256 or blake2b} of a given set of inputs
    \item \textbf{STORSRAM}: store a block into fast memory
    \item \textbf{LOADSRAM}: load a block from fast memory
    \item \textbf{STORDRAM}: store a block into slow memory
    \item \textbf{LOADDRAM}: load a block from slow memory
  \end{itemize}

  Any algorithm under this model is a sequence of instructions mentioned above. The first instruction takes time~$T$, the second
  and the third instructions takes negligible time. For the simplification of our analysis, we consider that the fourth instruction
  is executed whenever a new block is computed. The fifth instruction takes time $c \cdot T$.
\end{definition}

Note that this model can be easily modified to model computation on circuits without slow memory (the model previous analysed),
just disable the fourth and the fifth instruction and we are done.

\subsection{Memory hardness in the sequential setting}
In this section we define memory hardness in the sequential setting. The idea behind this definition is that such property of a function
is desirable: the majority of the complexity of the optimal algorithm (in terms of time complexity) will be spent on memory fetching,
which means that no mater how fast the computational module is, as long as the memory module is stable (not having rapid development),
then the development in faster computational module will not result in considerable boost in execution time. If such property is real,
then we can construct, say, a public block-chain using this function as its proof-of-work consensus method, without having to worry about
custom circuit will make computational power overly centralized (a problem severely affecting bitcoin.)

% This definition surly needs refurnishing, but I have no Idea how to use it without having to figure out the
% optimal solution now. I want to first assert that the optimal solution will have T_fetch < c_0 * T, then use
% that there exist a better solution, to prove that such solution is not optimal, and thus to prove the
% best solution (or algorithm) is indeed fetch taking, and thus memory-hard.

\begin{definition}\label{def::memoryHard}% after refinement, make a clearer definition
  Fix a function $f$, then it is considered to be memory-hard in a computational machine $\mathbb{M}$ in definition~\ref{def::dramComputationalModel}
  if given the optimal algorithm that can correctly output the result, the ratio of the time taken to fetch blocks from DRAM (denote as $T_f$) over
  the time taken to compute underlying hash function (denote as $T_h$) is greater than a constant $c_0$.
\end{definition}

The idea behind this definition is similar to Amdahl's law~\cite{amdahl1967validity}. In particular, even if the hashing hardware has evolved into
a level where all the hashes summed up in one execution of the function instance can be viewed as negligible, then the performance boost is
still bounded by a constant $1/c_0$. Although we currently are not aware how the optimal strategy are like given common graphs, the analysis in
section~\ref{sec::sandwich} shows that for random sandwich graph, for the optimal strategy, the ratio of fetching time $T_f$ over hashing time $T_h$
must satisfy a certain lower bound.

\subsection{Cumulative Pebbling Complexity in DRAM model}
In this section,we define New Cumulative Pebbling Complexity in DRAM model.In the Alwen's article,it introduces a new more fine-grained cost measure called the cumulative complexity (CC) of a graph. Put simply the CC of a given execution in the (parallel) pebbling game for graph G is the sum of the number of pebbles lying on G when summed across all steps in the execution.We change a lot in order to use in our DRAM model,next we will give the definition about them:

\begin{definition}\label{def::cumulativecomplexity}
Let $G$ be a $DAG$, P = ($P_0$,...,$P_t$) be an arbitrary pebbling of $G$ and $\Pi$ be the set of all complete pebblings of $G$. Then the (cumulative) cost of $P$ and then cumulative complexity ($CC$) of $G$ are defined respectively to be:\\
  \begin{equation}
    p-cost(P):= \sum\limits_{i=0}^t|P_i|
  \end{equation}

  \begin{equation}
    CC(G):= min\{p-cost(P):P \in \pi \}
  \end{equation}
\end{definition}

we suppose that the times which we fetch the block are $c$ times higher than do once hash.At the same time, to make our next analysis become easier and we also suppose that we don't make other work when we fetch the block in the DRAM .
\begin{definition}\label{def::cumulativecomplexity}
based on our this suppose ,we define the status when we fetch the block are $Q_i(i=1,\dots,s)$.So in DRAM model we have our definition of New Cumulative Pebbling Complexity in DRAM model:
  \begin{equation}
    p'-cost(P):= \sum\limits_{i=0}^t|P_i|+(c-1)\sum\limits_{i=0}^s|Q_i|
  \end{equation}

  \begin{equation}
    CC'(G):= min\{p'-cost(P):P \in \pi \}
  \end{equation}
\end{definition}


Armed with the definition above, we are ready to prove the time reduction in some proved
