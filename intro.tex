\paragraph{MHF's intuition} Memory hard hashing function has been widely applied as password hashing functions and as proof-of-effort functions. The reason why the
quality of memory hard\footnote{for concrete definitions, see definition~\ref{def::memoryHard}} is preferable is the asymmetric gap between
the computing power of commodity server (which the majority of honest users will use), and integrated circuits of different architectures (like ASIC,
FPGA, and GPU.) More specifically, consider that a malicious user trying to brute force every possible input combination in order to obtain the log-in
password after obtaining the hashed password files from a compromised authentication server, or one who tries to use custom circuit to compute proof-of-%
effort functions. In either settings, the inputs are irrelevant from each other (we consider the most naive brute force attack), and when this is
the case, features like parallelism and pipelining can help custom circuits beats computers of common architecture. The authors of \cite{corrigan2016balloon}
claims that it takes 100000 times more energy to compute a SHA-256 hash on a
general-purpose x86 CPU than it does to compute SHA-256 on special-purpose hardware. The advantage lies that when the input are completely independent,
the pipeline can hardly be disturbed (recall that modern CPUs do a lot of work to prevent this from happening, like when a bench command is encountered.)
And thus such circuits can achieve as high as one hash per clock cycle.

The intuition behind memory hard functions is that although the computing power of custom circuits have taken absolute advantage, constructing enough memory
on VLSI is a technical challenge not yet solved. Two main types of memory in widely use nowadays are DRAM and SRAM. % reference needed
DRAM is relatively cheap and has large volume, but are considerably slow compared to modern CPU, not to mention custom circuits.
SRAM, on the other hand, is fast % how fast?
, but every single memory cell in SRAM takes more transistors than that in DRAM, and thus a SRAM circuit takes up more space and energy that DRAM circuit of the
same volume. Design of custom circuit usually use SRAM on board in order to match the speed of other components, while in commodity servers, DRAM and cache made
of SRAM are used in combination. And therefore computing any function on custom circuits and on commodity server are limited to two strategies:

\begin{itemize}
  \item Use custom circuit, which has faster computation speed and limited memory
  \item Use commodity server, which has slower computation speed and large memory (although slow)
\end{itemize}

Intuitively, if a function has the property that when working memory is smaller than a particular threshold, then the time to finish the computation
becomes considerably longer. In particular, given concrete parameter settings, if such increase in time makes computational time on custom circuits
greater than or equal the time honest users would take, then such memory hard property will render the advantage of custom circuits useless.

\subsection{Theoretical analysis}
The analysis of memory hard functions usually use a ``graph reduction'' method in order to simply the analysis process. That is, after defining a pebbling
rule on a DAG (the data dependency graph of the MHF), any valid algorithm computing the function can be converted (with high probability) to pebbling on
that graph. And by graph labeling\cite{dziembowski2011one}, any legal graph pebbling can be converted to an algorithm computing the function. And thus
in order to lower bound the time to compute the function, one only needs to lower bound the steps to pebble the underlying DAG. Some notable functions include
Balloon hashing, scrypt, argon2, catena, and a function proposed by Alwen et.al. from \cite{alwen2015high} that is proved to enjoy memory hard property
under a parallel setting.

\paragraph{Balloon hashing}
Balloon hashing\cite{corrigan2016balloon} is a hashing function proposed by Dan Boneh et.al.
With the application of password hashing algorithm in mind, the algorithm was designed to
have three important properties:

\begin{itemize}
  \item \textit{Memory hardness} product of time ($T$) and working space ($S$) is proved to be greater than $n^2/32$
  \item \textit{Password independent} the memory access pattern is independent of the password being hashed.
  \item \textit{Performance} the Balloon algorithm is easy to implement and it matches or exceeds the performance of the fastest comparable password-hashing algorithms
\end{itemize}

The sparking point of the analysis of balloon hashing is that it avoids asymptotic notations whenever possible, and thus under the assumptions the
exact lower bound of the function can be found.

\paragraph{Work of Alwen et.al.}
An impressing research line of Alwen et.al. focused on extending the analysis of memory hard function to a parallel setting and device attacks on known sequential
memory hard functions. In particular, they designed general attack utilising parallelism.

\subsection{Our contribution}
In this work, we presented a new model for computing the memory hard functions. Under the assumption of this model (which is considered valid under
current electronic technology), the lower bound of known memory hard functions can be lowered. More specifically, we can lower the $ST$ bound of balloon
hashing from $r \cdot n^2/32$ to $c \cdot r \cdot n$, and reduce the cumulative complexity $CC(G)$ of the graph constructed from \cite{alwen2015high}
from $\Omega(n^2 \cdot \log^3(n))$ to roughly $O(n \cdot \log^3(n))$.
